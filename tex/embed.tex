\chapter{Embedding}
Because in general programs compiled to Ising graphs can be of any shape, we need a way to convert arbitrary graphs into Chimera graphs that can be executed on a machine.  This is done by adding \emph{clone spins} into the graph: each clone spin has the same value in the ground state as one of the logical spins in the input graph.  Thus we can decrease the connectivity of the input graph until it is sparse enough to be isomorphic to a physical machine.  We expect that translating complete graphs onto graphs with constant neighbour counts to require roughly a quadratic size penalty in the number of clone spins (as the number of connection in a complete graph $~ n^2$).
Ideally, to ensure that each clone spin's ground state is the same as it's parent we would have $J_{ij} << J_{clone} \forall i,j$.  Because of physical limits of the machine (especially the resolution mentioned above) large values of $J_{clone}$ may not maximize fidelity: for a machine with finite resolution, having a large $J_{clone}$ may compress other coupling values together such that the ground state starts to overlap with higher energy state.  On the other hand a too small value of $J_{clone}$ may result in ground states in which the clones are not aligned, which will almost certainly also produce incorrect ground states. Thus we have to find an empirical value of $J_{clone}$ that maximizes the fidelity while preserving the ground states.

\section{Embedding algorithm}

The embedding algorithm is as follows:

\emph{\textbf{Definitions:}}

Designate the input Hamiltonian graph $V$ and the output Chimera graph $G$.  Label the spins in $V$ and $G$ as $V_i$ and $G_i$ respectively.
We define the \emph{clone map} $C_i$ as the set of spins $[i,j \ldots]$ which have the same logical value as their parent spin: so for example the clone map of spin 5, $C_5$, might be $[2,3,13]$ which would mean that spins 2, 3, 5 and 13 all share the same logical value.  
We define a mapping $M$ between logical spins in graph $V$ and computation spins in graph $G$.  That is, spin $V_{12}$ representing the 12th variable in a SAT problem might be mapped to $G_{187}$, the 187th machine spin.

We also define a \emph{clone coupling} value which is as large as possible and ferromagnetic; this attempts to ensure that all the members of a clone map are aligned in the ground state.  In the final $G$, each member of a clone map should have a clone coupling connecting it to at least one other member of the clone map.

\emph{\textbf{Procedure:}}
\begin{enumerate}
	\item Populate $M$ by mapping each $V_i$ to one of the spins $G_j$ on the left side of a qubyte that lies along the diagonal of $G$
	\item For each field term in $V$, add a field to $G$ on the corresponding spin
	\item For each $J_{ij}$ in $V$, conduct the following operation:
		\begin{itemize}
			\item Scan through both of $C_i$ and $C_j$ to find the pair which are nearest to each other in $G$; call these $x$ and $y$
			\item Get a list of each spin that lies along the path between $x$ and $y$
			\item Assign half of these spins to $C_i$ and half to $C_j$; add the appropriate clone coupling into $G$ for each spin in the path to ensure that the clone map is properly connected
			\item Finally, at the interface between the two new clone map members, add a coupling into $G$ with the same value as $J_{ij}$
		\end{itemize}
\end{enumerate}

Each term in $V$, both fields and couplings, should now have a corresponding term in $G$.  $G$ should also contain many coupling terms that group the necessary clone maps.  Figure \ref{fig:embedding} shows an example of the embedding process.


\begin{figure}
	%\scalebox{}{
	%	\includegraphics[]{}
	%}
	\caption[Embedding Algorithm]{This figure shows three steps of the embedding algorithm: mapping spins from $V$ to $G$, tracing the path for the clones between $V_i$ and $G_j$ and partitioning the path into $C_i$ and $C_j$.}
	\label{fig:embedding}
\end{figure}

This algorithm is tailored AQC machines whose qubits are arranged in a square shape (specifically the Chimera graph), however, the principle should be easily generalizable: as mentioned above any reasonably compact graph with constant connectivity should be able to embed a complete graph with an $n^2$ overhead.
