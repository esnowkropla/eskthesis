\chapter{Boolean Satisfiability}
In this chapter we describe the NP-Complete problem \emph{boolean satisfiability} (SAT) and the techniques needed to use it as a problem for benchmarking the \machine.

\section{Satisfiability Problems}
Boolean satisfiability is the problem of determining if there exists an assignment of boolean variables to satisfy a given boolean formula.  A boolean formula is \emph{satisfiable} if there is some assignment of the variables within it that causes the whole expression to evaluate to \texttt{true}, and is \emph{unsatisfiable} if there is no such assignment.  
A boolean expression in conjunctive normal form consists of the conjunction of a set of clauses, each clause consisting of the disjunction of some number of variables (or their negations).  When each clause has $k$ variables, then the problem of finding a satisfying assignment is called $k$-SAT.  An example 3-SAT expression might look like the following

\begin{equation}
	(x_1 \vee x_2 \vee \neg x_3) \wedge (\neg x_1 \vee \neg x_2 \vee \neg x_3) \wedge (x_1 \vee \neg x_2 \vee x_3)
\end{equation}

which has five satisfying assignments:

\begin{center}
\begin{tabular}{l || l | l | l | l | l}
	\multicolumn{1}{l}{} & \multicolumn{5}{c}{Assignments} \\
	$x_1$ & \texttt{T} & \texttt{T} & \texttt{T} & \texttt{F} & \texttt{F} \\
	$x_2$ & \texttt{F} & \texttt{F} & \texttt{T} & \texttt{T} & \texttt{F} \\
	$x_3$ & \texttt{F} & \texttt{T} & \texttt{F} & \texttt{T} & \texttt{F} \\
\end{tabular}
\end{center}

$k$-SAT is NP-Complete\cite{sat} for $k > 2$, and is interesting because it was the first and is one of the most general NP-Complete problems; it is generally very easy to find reductions from other problems into boolean satisfiability.  For this reason we use boolean satisfiability (specifically 3-SAT) for the testing of the \machine machine.

\section{Randomly Generating 3-SAT Problems}
We can build 3-SAT instances with $N$ variables and $M$ clauses by randomly selecting three elements from the set of $N$ variables and their negations $M$ times.  Care must be taken when choosing $N$ and $M$ however; while 3-SAT is NP-Complete in general, the difficulty of any particular instance depends strongly on the ratio $\alpha = M/N$\cite{Kirkpatrick}.  In the small $\alpha$ regime there are many satisfying assignments, and any algorithm to find a solution will likely stumble on one by chance.  Contrariwise, in the large $\alpha$ regime, there are likely to be \emph{no} satisfying assignments.

As discussed in Kirkpatrick and Selman\cite{Kirkpatrick}, there is a phase transition between the many solution and zero solution regimes, which occurs around the value of $\alpha$ for which there is likely to be one solution.  This is the region where finding a satisfying assignment (or proving there is none) is the hardest, and helps to keep us from falsely believing our algorithm is better than it is due to too easy SATs.  This transition is sharper for larger values of $N$ or $M$, and broader for smaller values.  As the value of $N$ and $M$ that will fit onto the \machine machine are very small, we don't expect to see a very sharp difference for different values of $\alpha$.  Kirkpatrick and Selman report that for 3-SAT, the transition occurs at $\alpha \sim 4.17$.  Therefore we shall attempt to generate SATS which lie as close to this transition as possible; in practice this usually means around $4.25 \pm 0.25$ due to the fact that $N$ and $M$ must be integers and the size of our hardware graph restricts us to small values of $N$ and $M$.

\section{Embedding 3-SAT}
Embedding 3-SAT problems requires two different uses for the qubits: each variable in the SAT requires a qubit, and each clause also requires a qubit.  This is because we embed each clause as a \texttt{3OR} gate, a 4 spin gate where the fourth spin's logical value is the disjunction of the three input spins.  This gate requires an auxiliary spin beyond the computational spins to encode its logic.  Because we know that for a satisfying statement each clause must be satisfied, clamping each clause's output variable completes the 3-SAT \texttt{QSM} construction.  As we have roughly four times as many clauses as variables, each clause shares its variables with many other clauses.  This means that the resulting graph is reasonably close to being a complete graph, so we embed them as if they were and tolerate some inefficiency. 

Since the largest complete graph the hardware graph can embed is 32 spins, the largest SAT problem we could embed at an $\alpha$ close to the phase transition is 6 variables and 25 clauses for an alpha of 4.17; however, as discussed in Section \ref{sec:inop_qubits}, the capability of the \machine machine during our experiments was actually 29 spins, which is too small for 6 SAT variables plus 25 clauses.  The actual largest SAT embedded was thus 6 SAT variables and 21 clauses.

\section{Example SAT embedding}

