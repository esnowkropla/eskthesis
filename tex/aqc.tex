\chapter{Adiabatic Quantum Computing}
\section{Literature Survey}
Introduced by Farhi et.\ al.\cite{farhi}, the idea of adiabatic quantum computation (hereafter AQC) is to exploit the adiabatic theorem to solve computational problems.  There are two main components to the idea.  First, we find a Hamiltonian such that the ground state is the solution to a computational problem (e.g. a bitstring of spins pointed up and down).  Second, we use the adiabatic theorem to move from some easily prepared initial state into the Hamiltonian we found.

\section{Finding a Problem Hamiltonian}
While in principle there are an unlimited number of ways to construct a Hamiltonian whose ground state encodes the solution to a computation, our method is to use an N-particle Hamiltonian of the form

\begin{displaymath}
	H = \sum_{i} h_i \sigma_i^Z + \sum_{i < j} J_{ij} \sigma_i^Z\sigma_j^Z 
\end{displaymath}
where $\sigma_i^Z$ is the pauli matrix of the ith particle and $h_i$ and $J_{ij}$ are the parameters of the Hamiltonian.  This 2-local Hamiltonian corresponds to a graph structure, where each particle is a vertex and each non-zero $J_{ij}$ is an edge, while non-zero $h_i$s can be represented as edges to a constant "field" spin.

Our problem of finding a suitable Hamiltonian is now reduced to finding a set of $h$'s and $J_{ij}$'s such that our desired solution is encoded in the ground state.  We do this by breaking our problem down into sub-problems and finding Hamiltonians for these easier sub-problems, and then reassembling these smaller Hamiltonians into the final problem Hamiltonian using the gluing theorem.\cite{gluing}

Figure FIXME shows a Hamiltonian graph describing the $h_i$'s and $J_{ij}$'s to encode the logic of a NAND gate.  Because NAND's are universal for computation, combining this graph with the gluing theorem allows us to encode the evaluation of any computable function into the ground state of a Hamiltonian of the general form we described above.  

We don't expect this encoding using NANDs glued together to be either optimal in the sense of using fewest spins or couplings, or to be asymptotically faster than an equivalent classical circuit.  We have no general solution for the first problem; each computational problem we want to encode efficiently requires it's own bespoke encoding.  To solve the second problem we take advantage of the fact that quantum mechanics is time-reversible: which variables are output and which are input is arbitrary.  Thus if we have a solution in mind for a given problem we can simply encode the Hamiltonian ``backwards'' and recover answers that would lead to our initial solution.  For NP problems, where verifying a solution is in P, we can thus write a verification circuit with a solution of \texttt{true} and find the inputs to our NP problem.

Our approach to solving NP problems is thus:
\begin{itemize}
	\item Construct a circuit to verify a candidate solution (for satisfiability this would be evaluation of the clauses; for factoring this would be a multiplication circuit)
	\item Clamp the ``output'' spins to their required values (for a satisfiability problem this would be all output spins \texttt{true}; for a factoring problem this the number to factor)
	\item Run the adiabatic computer and read off the ``input'' spins
\end{itemize}

\section{Adiabatic Evolution}
