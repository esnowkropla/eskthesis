\chapter{Conclusion}
A variety of Hamiltonians were constructed and evaluated on the \texttt{VESUVIUS} hardware.  We have seen several general methods for embedding problems into a form suitable for evaluation on an adiabatic quantum computer at a modest overhead, although even that overhead is a stiff price to pay on the current generation of adiabatic hardware.  Although the machine was not able to solve a 6 variable SAT problem, it was capable enough to solve several different toy problems.

In the process of solving these toy problems we discovered several factors important for using the D-Wave family of adiabatic machines.  Among them the fact that contrary to expectations shorter anneal times give better results, and that the limited number of built in machine couplings requires careful thought when embedding to ensure that coupling collisions do not occur.
