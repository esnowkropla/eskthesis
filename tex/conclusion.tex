\chapter{Conclusion}
A variety of Hamiltonians were constructed and evaluated on the \machine hardware.  We have seen several general methods for embedding problems into a form suitable for evaluation on an adiabatic quantum computer at a modest overhead, although even that overhead is a stiff price to pay on the current generation of adiabatic hardware.  While the machine is able to solve single clause SAT problems and other small (8 qubit) problems, a real (still small) problem of 6 SAT variables was too difficult. The machine \emph{was} able to solve a similar number of qubits when the SAT was simple (the ``k44\_and'' case), so the difficulty of the 6 variable SAT must not be purely based on the number of qubits.  

In the process of solving these problems we discovered several factors important for using the D-Wave family of adiabatic machines.  Among them the fact that contrary to expectations shorter anneal times give better results, and that the limited number of built in machine couplings requires careful thought when embedding to ensure that coupling collisions do not occur.
