\chapter{Introduction}

\section{Classical Computing}

\section{Feynman to Shor}

\section{The Adiabatic Theorem}
The adiabatic theorem states that if a system is initially prepared in the $ith$ energy level, and the Hamiltonian is evolved according to the adiabatic condition, the system will remain in the $ith$ state after the evolution.  The adiabatic condition is given as:

\begin{displaymath}
	\left | \frac{1}{\omega_{fi}}\frac{\partial}{\partial t} \braket{\psi_f | \hat{V}(t) | \psi_i} \right | << |E_f - E_i|
\end{displaymath}

where $E_i$ and $E_f$ are the energies of the initial and final states, $\hat{V}(t)$ is the time dependant part of the Hamiltonian and $\omega_{if} = \frac{E_f - E_i}{\hbar}$ is a convenient variable.  This (FIXME missing!) derivation due to \cite{zettili}.
