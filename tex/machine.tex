\chapter{The \texttt{VESUVIUS} Machine}

\section{Machine Variables}
The \texttt{VESUVIUS} machine on which our experiments were conducted is composed of 512 rf-SQUID superconducting qubits of the type described in \cite{qubit} \cite{dwave_nature}. It is the same phsyical machine as the larger of the two machines described in \cite{pudenz}.
As mentioned in Chapter \ref{chap:aqc} the form of the Hamiltonian we use for AQC is

\begin{displaymath}
	H_{tot} = f(t)H_i + g(t)H_f
\end{displaymath}

and the experiments were carried out with a transverse field of $f(0) \approx 33.8 $ GHz and a maximum normalized coupling value of $g(T) \approx 20.5$ GHz, and an annealing temperature of 17 mK.  These correspond to energies of $140 \mu$ eV and $85 \mu$ eV for the Hamiltonian components respectively and a thermal energy of $1.5 \mu$ eV.

\section{Resolution}
In addition to the graph-shape restrictions, the \texttt{VESUVIUS} machine cannot implement arbitrary physical Hamiltonians.  The machine can only implement fields and couplings as one of 15 distinct values: $-7/7, -6/7 \dots 5/7,6/7, 7/7$.  Call the largest coupling present in the problem Hamiltonian $C_{max}$, and the smallest \emph{magnitude} coupling $C_{min}$.  As our embedded Hamiltonians don't generally have a $C_{max}$ to $C_{min}$ distance of exactly 7, they must be manipulated to fit in this range.
First  all the field and coupling values are normalized into the range $[-1..1]$, and then each coupling is coerced to the nearest machine implemented value.
Unfortunately we don't know exactly how the coercion is done; it seems reasonable that each normalized coupling would be rounded to the nearest machine implemented coupling, but other schemes such as rounding toward or away from zero are possible.
The linear spacing of the machine implemented couplings implies that Hamiltonians with couplings that linearly fill the range from $-C_{max}$ to $C_{max}$ will be least affected by the coercing issue, while those Hamiltonians which have gaps between coupling values or other spacings will be most impacted by the effects of finite machine resolution.

\section{Programming Noise}
The resolution issue is exacerbated by the fact that there is some error in the programming of each coupling (or field).  The magnitude of this error is uncertain; efforts are underway to quantify how much error is consistent with results from the machine \cite{aaron}.
The preliminary results in Chapter \ref{chap:prelim} suggest that the programming noise is in the neighbourhood of the machine coupling spacing, i.e. 1/7 of $g(t)$.
