\documentclass{article}
\usepackage{amsmath, url, listings, graphicx}
\title{Adiabatic Quantum Computation}
\author{Elliot Snow-Kropla\\Kyriakidis Group}
%\date{20 July 2012}

\lstset{ %
	tabsize = 2,
	frame = single,
	basicstyle = \ttfamily,
	escapebegin = <\>
}

\begin{document}
\maketitle

\subsection{Embedding}
Because in general programs compiled to Ising graphs can be of any shape, we need a way to convert arbitrary graphs into Chimera graphs that can be executed on a machine.

\emph{ Embedding algorithm}

The embedding algorithm is as follows:

\begin{enumerate}
	\item Designate the input graph $V$ and the destination Chimera graph $G$
	\item Label the spins in the input graph $V_i$
	\item Designate the cloning map from a given logical spin to its set of clones $G_i \mapsto [G_0 \ldots G_n] $ Initially, the
		clone list will contain only the parent logical spin
	\item Assign each spin $V_i$ to a spin in the left side of a $K_{4,4}$ that lies along the diagonal of $G$
	\item For each edge $V_{i,j}$ in $V$, iterate pairwise through the clone sets of $G_i,G_j$ until we find the pair $G_x,G_y$ such that
		the path from $G_x$ to $G_y$ is minimal
	\item Divide the spins in the path into two groups at the first kink in the path: assign all the spins in one group to 
\end{enumerate}

\bibliography{mybib}{}
\bibliographystyle{plain}
\end{document}
