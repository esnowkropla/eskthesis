\documentclass{article}
\usepackage{amsmath, url, listings, graphicx}
\title{Discrete Logarithm Circuits}
\author{Elliot Snow-Kropla\\Kyriakidis Group}
%\date{20 July 2012}

\lstset{ %
	tabsize = 2,
	frame = single,
	basicstyle = \ttfamily,
	escapebegin = <\>
}

\begin{document}
\maketitle
The most straightforward way to make an AQC circuit for a one-way function is to take advantage of the fact that out computations are reversible; thus we just make a circuit for computing the function going the easy way.

\section{Circuits}
The easy direction for the discrete log problem is modular exponentiation, $c = b^e \text{ (mod }N\text{)}$.  Based on the repeated squares technique for exponentiation, we use the following technique for modular exponentiation\cite{mod_exp}: write $e$ in binary as 


\begin{lstlisting}
def modular_power(b, e, m):
	r = 1
	while e > 0:
		if (e % 2) == 1:
			r = (r * b) % m
		e = e >> 1
		b = (b * b) % m
	return r
\end{lstlisting}



\bibliography{mybib}{}
\bibliographystyle{plain}
\end{document}
